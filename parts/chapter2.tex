\chapter{Technical background}
This chapter presents an overview of the technical concepts involved in the realization of this project, namely the Ethereum Blockchain and Web applications
\section{Ethereum Blockchain}
Ethereum is a blockchain platform with its own \gls{cryptocurrency}, called Ether (ETH) or Ethereum, and its own programming language, called Solidity. As a blockchain network, Ethereum is a decentralized public ledger for verifying and recording transactions. The network's users can create, publish, monetize, and use applications on the platform, and use its Ether cryptocurrency as payment. Insiders call the decentralized applications on the network "dapps."

\begin{wrapfigure}[10]{r}{4cm}
	\vspace{-10pt}
	\includegraphics[width=4cm]{images/chapter2/ethereum.png}
	\vspace{-10pt}
	\caption{{\footnotesize Ethereum Blockchain logo}}
\end{wrapfigure}

The intent of Ethereum is to create an alternative protocol for building decentralized applications, providing a different set of tradeoffs that its creators believe will be very useful for a large class of decentralized applications, with particular emphasis on situations where rapid development time, security for small and rarely used applications, and the ability of different applications to very efficiently interact, are important. Ethereum does this by building what is essentially the ultimate abstract foundational layer: a blockchain with a built-in Turing-complete programming language, allowing anyone to write smart contracts and decentralized applications where they can create their own arbitrary rules for ownership, transaction formats and state transition functions.

In the Ethereum universe, there is a single, canonical computer (called the Ethereum Virtual Machine, or \acrshort{EVM}) whose state everyone on the Ethereum network agrees on. Everyone who participates in the Ethereum network (every Ethereum node) keeps a copy of the state of this computer. Additionally, any participant can broadcast a request for this computer to perform arbitrary computation. Whenever such a request is broadcast, other participants on the network verify, validate, and carry out ("execute") the computation. This causes a state change in the EVM, which is committed and propagated throughout the entire network.

\subsection{Accounts}
In Ethereum, the state is made up of objects called "accounts", with each account having a 20-byte address and state transitions being direct transfers of value and information between accounts. An Ethereum account contains four fields:

\begin{itemize}
\item The \textbf{nonce}, a counter used to make sure each transaction can only be processed once
\item The account's current \textbf{ether} balance
\item The account's \textbf{contract code}, if present
\item The account's \textbf{storage} (empty by default)
\end{itemize}

"Ether" is the main internal crypto-fuel of Ethereum, and is used to pay transaction fees. In general, there are two types of accounts: \textbf{externally owned accounts}, controlled by private keys, and \textbf{contract accounts}, controlled by their contract code. An externally owned account has no code, and one can send messages from an externally owned account by creating and signing a transaction; in a contract account, every time the contract account receives a message its code activates, allowing it to read and write to internal storage and send other messages or create contracts in turn.\newpage

\begin{figure}[h]
	\centering
		\includegraphics[width=10cm]{images/chapter2/accounts.png}
		\caption{{\footnotesize Abstract Ethereum accounts visualization}}
\end{figure}

\subsection{Transactions}
An Ethereum transaction refers to an action initiated by an externally-owned account, in other words an account managed by a human, not a contract. For example, if Bob sends Alice 1 ETH, Bob's account must be debited and Alice's must be credited. This state-changing action takes place within a transaction.

\begin{figure}[H]
	\centering
		\includegraphics[width=10cm]{images/chapter2/transition.png}
		\caption{{\footnotesize Abstract representation of an Ethereum transaction}}
\end{figure}

Transactions, which change the state of the EVM, need to be broadcast to the whole network. Any node can broadcast a request for a transaction to be executed on the EVM; after this happens, a miner will execute the transaction and propagate the resulting state change to the rest of the network.

Transactions require a fee and must be mined to become valid. To make this overview simpler we'll cover gas fees and mining elsewhere.

A submitted transaction includes the following information:

\begin{itemize}
\item \textbf{recipient} - the receiving address (if an externally-owned account, the transaction will transfer value. If a contract account, the transaction will execute the contract code)
\item \textbf{signature} - the identifier of the sender. This is generated when the sender's private key signs the transaction and confirms the sender has authorised this transaction
\item \textbf{value} - amount of ETH to transfer from sender to recipient (in WEI, a denomination of ETH)
\item \textbf{data} - optional field to include arbitrary data
\item \textbf{gasLimit} - the maximum amount of gas units that can be consumed by the transaction. Units of gas represent computational steps
\item \textbf{gasPrice} - the fee the sender pays per unit of gas
\end{itemize}

Gas is a reference to the computation required to process the transaction by a miner. Users have to pay a fee for this computation. The gasLimit and gasPrice determine the maximum transaction fee paid to the miner (more on gas on later subsections.

\subsection{Gas}
Gas refers to the unit that measures the amount of computational effort required to execute specific operations on the Ethereum network.

Since each Ethereum transaction requires computational resources to execute, each transaction requires a fee. Gas refers to the fee required to successfully conduct a transaction on Ethereum.

\begin{figure}[h]
	\centering
		\includegraphics[width=10cm]{images/chapter2/gas.png}
		\caption{{\footnotesize Diagram of the role of gas in Ethereum \cite{TakenobuhsEthereumevmillustratedEthereum}}}
\end{figure}

In essence, gas fees are paid in Ethereum's native currency, ether (ETH). Gas prices are denoted in gwei, which itself is a denomination of ETH - each gwei is equal to 0.000000001 ETH (10-9 ETH). For example, instead of saying that your gas costs 0.000000001 ether, you can say your gas costs 1 gwei.

Gas fees exist because they help keep the Ethereum network secure. By requiring a fee for every computation executed on the network, we prevent actors from spamming the network. In order to prevent accidental or hostile infinite loops or other computational wastage in code, each transaction is required to set a limit to how many computational steps of code execution it can use. The fundamental unit of computation is "gas".

\subsection{Consensus mechanisms}
When it comes to blockchains like Ethereum, which are in essence distributed databases, the nodes of the network must be able to reach agreement on the current state of the system. This is achieved using consensus mechanisms.

Consensus mechanisms (also known as consensus protocols or consensus algorithms) allow distributed systems (networks of computers) to work together and stay secure.

For decades, these mechanisms have been used to establish consensus among database nodes, application servers, and other enterprise infrastructure. In recent years, new consensus protocols have been invented to allow cryptoeconomic systems, such as Ethereum, to agree on the state of the network.

A consensus mechanism in a cryptoeconomic system also helps prevent certain kinds of economic attacks. In theory, an attacker can compromise consensus by controlling 51\% of the network. Consensus mechanisms are designed to make this "\gls{51attack}" unfeasible. Different mechanisms are engineered to solve this security problem differently.
\subsubsection{Types of consensus mechanisms}

\begin{description}
\item[Proof of work] Proof-of-work is done by miners, who compete to create new blocks full of processed transactions. The winner shares the new block with the rest of the network and earns some freshly minted \gls{ETH}. The race is won by whoever's computer can solve a math puzzle fastest – this produces the cryptographic link between the current block and the block that went before. Solving this puzzle is the work in "proof of work".
\item[Proof of stake] Proof-of-stake is done by validators who have staked ETH to participate in the system. A validator is chosen at random to create new blocks, share them with the network and earn rewards. Instead of needing to do intense computational work, you simply need to have staked your ETH in the network. This is what incentivises healthy network behaviour.
\end{description}

\subsection{Dapps}

A dapp has its backend code running on a decentralized peer-to-peer network. Contrast this with an app where the backend code is running on centralized servers.

A dapp can have frontend code and user interfaces written in any language (just like an app) that can make calls to its backend. Furthermore, its frontend can be hosted on decentralized storage.

\begin{description}
\item[Decentralized] means they are independent, and no one can control them as a group.
\item[Deterministic] they perform the same function irrespective of the environment they are executed.
\item[Turing complete] which means given the required resources, the dapp can perform any action.
\item[Isolated] which means they are executed in a virtual environment known as Ethereum Virtual Machine so that if the smart contract happens to have a bug, it won’t hamper the normal functioning of the blockchain network.
\end{description}

\begin{figure}[H]
	\centering
		\includegraphics[width=10cm]{images/chapter2/dapps.png}
		\caption{{\footnotesize Diagram of centralized and decentralized applications topology}}
\end{figure}

\subsection{Smart contracts}

A smart contract is simply put a program that runs on the Ethereum blockchain. It's a collection of code (its functions) and data (its state) that resides at a specific address on the Ethereum blockchain.They runs exactly as programmed. Once they are deployed on the network they can't be changed. Dapps can be decentralized because they are controlled by the logic written into the contract, not an individual or company.

Smart contracts are a type of Ethereum account. This means they have a balance and they can send transactions over the network. User accounts can then interact with a smart contract by submitting transactions that execute a function defined on the smart contract. Smart contracts can define rules, like a regular contract, and automatically enforce them via the code.\medskip

Ethereum has developer-friendly languages for writing smart contracts:

\begin{itemize}
\item Solidity
\item Vyper 
\end{itemize}

\section{Web Applications}
\subsection{Structure}
\subsection{Business use}
\subsection{Development}